%
% Qualificacao_Doutorado.tex (LateX)
% 
% Objetivo: Arquivo principal do relatório de qualificação de doutorado.
% baseado em um template para a geração de documentos em LaTeX.
% 
% Versão 1.0
% 
% Site: http://www.dirackslounge.online
% 
% Programador: 
%		(1.4) - Lauro César Araujo 
%			Template, distribuição e manutenção
%		(1.5) - Rodolfo A. C. Neves (Dirack) 07/10/2019 
%			Modificações e utilização neste relatório
% 
% Email: rodolfo_profissional@hotmail.com
% 
% Licença (Versão modificada): GPL-3.0 <https://www.gnu.org/licenses/gpl-3.0.txt>.
%
% Licença (Versão original): LaTeX Project Public License (LPPL - 1.3) <http://www.latex-project.org/lppl.txt>
%
% Documentação extra: <http://abntex2.googlecode.com/>

\documentclass[
	% -- opções da classe memoir --
	12pt,				% tamanho da fonte
	openright,			% capítulos começam em pág ímpar (insere página vazia caso preciso)
	oneside,			% para impressão em verso e anverso. Oposto a oneside
	a4paper,			% tamanho do papel. 
	% -- opções da classe abntex2 --
	%chapter=TITLE,		% títulos de capítulos convertidos em letras maiúsculas
	%section=TITLE,		% títulos de seções convertidos em letras maiúsculas
	%subsection=TITLE,	% títulos de subseções convertidos em letras maiúsculas
	%subsubsection=TITLE,% títulos de subsubseções convertidos em letras maiúsculas
	% -- opções do pacote babel --
	english,			% idioma adicional para hifenização
%	french,				% idioma adicional para hifenização
%	spanish,			% idioma adicional para hifenização
	brazil				% o último idioma é o principal do documento
	]{abntex2}

\usepackage{multirow}
\usepackage{amsmath}
\usepackage{tocloft}
\usepackage{cmap}			% Mapear caracteres especiais no PDF
\usepackage{lmodern}			% Usa a fonte Latin Modern			
\usepackage[T1]{fontenc}		% Seleção de códigos de fonte.
\usepackage[utf8]{inputenc}		% Determina a codificação utiizada (conversão automática dos acentos)
\usepackage{makeidx}            	% Cria o indice
\usepackage{amssymb,amsfonts,amsmath,wasysym}
\usepackage{lastpage}			% Usado pela Ficha catalográfica
\usepackage{indentfirst}		% Indenta o primeiro parágrafo de cada seção.
\usepackage{nomencl} 			% Lista de simbolos
\usepackage{color}			% Controle das cores
\usepackage{graphicx}			% Inclusão de gráficos
\usepackage{microtype} 			% para melhorias de justificação
\usepackage{subfig}
\usepackage{float} 			% Colocar a figura no local certo
\usepackage{scalefnt} 			% Pacote da redimensionar a fonte de tabelas, figuras e equações.
\usepackage{placeins}
\allowdisplaybreaks
\usepackage{kantlipsum}
\usepackage{pdfpages}
\usepackage{titlesec}
\usepackage[brazilian,hyperpageref]{backref}	 % Paginas com as citações na bibl
\usepackage[alf]{abntex2cite}			 % Citações padrão ABNT
\usepackage{tensor}
\usepackage{setspace}
\usepackage{caption}
\usepackage{amsmath}
\usepackage{enumerate}
\usepackage[portuguese,ruled,lined]{algorithm2e}
\usepackage{algorithmic}
\usepackage{url}

\renewcommand{\figurename}{Figura-}
\usepackage[figurename=Figura]{caption}

\renewcommand{\chapnumfont}{\bfseries}
\renewcommand{\ABNTEXfontereduzida}{\mdseries\footnotesize}
\renewcommand{\ABNTEXchapterfontsize}{\bfseries \normalsize}
\renewcommand{\ABNTEXsectionfontsize}{\normalsize}
\addto{\captionsbrazil}{\renewcommand{\contentsname}{\textbf{SUMÁRIO}}}
\addto{\captionsbrazil}{\renewcommand{\bibname}{\bfseries \textbf{REFERÊNCIAS}\selectfont}}
\renewcommand{\apendicesname}{\bfseries \selectfont  APÊNDICES}
\renewcommand{\apendicename}{\bfseries\selectfont APÊNDICE}
\renewcommand{\anexosname}{\bfseries ANEXOS}

\renewcommand*{\backrefalt}[4]{
	\ifcase #1
	
	\or
	
	\else

	\fi}

%% Informações básicas da CAPA
\instituicao{
  Serviço Nacional de Aprendizagem Comercial -- SENAC
  \par
  Tecnologia em Análise e Desenvolvimento de Sistemas
  \par
  Conceitos de Computação 1}
\titulo{A ARQUITETURA DE HARDWARE ARM}

\autor{Rodolfo André Cardoso Neves}
\local{Belém-Pará}
\data{2021}

\orientador{Prof. Anderson Aparecido Alves da Silva}
\coorientador{}

\preambulo{Produção Textual Individual (PTI) apresentada ao Programa de Tecnologia em Análise e Desenvolvimento de Sistemas do
Serviço Nacional de Aprendizagem Comercial (SENAC), em cumprimento às exigências da disciplina Conceitos de Computação 1.}

\definecolor{blue}{RGB}{41,5,195}
\definecolor{black2}{RGB}{39,64,139}

%% Configurações padrão do PDF
\hypersetup{
		backref=true,
		pagebackref=true,
		bookmarks=true,         		% show bookmarks bar?
		pdftitle={\imprimirtitulo}, 
		pdfauthor={\imprimirautor},
    	pdfsubject={\imprimirpreambulo},
		pdfkeywords={PALAVRAS}{CHAVES}{abnt}{abntex}{abntex2},
	    pdfproducer={LaTeX with abnTeX2}, 		% producer of the document
	    pdfcreator={\imprimirautor},
    	colorlinks=true,       				% false: boxed links; true: colored links
    	linkcolor=black,          			% color of internal links
    	citecolor=black,        			% color of links to bibliography
    	linkcolor=black,          			% color of internal links
    	citecolor=black,        			% color of links to bibliography
    	filecolor=black,      				% color of file links
		urlcolor=black,
		bookmarksdepth=4
}

% Indentação do parágrafo
\setlength{\parindent}{1.3cm}

% Espaçamento entre parágrafos
\setlength{\parskip}{0.2cm}

% Espaçamento entre linhas
\OnehalfSpacing	

% compilar indice
\makeindex

% Compilar lista de abreviaturas e siglas
\makenomenclature

\renewcommand{\sin}{\mathrm{sen}}
\newcommand{\disp}{\displaystyle}
\newcommand{\mbf}{\mathbf}

\hyphenation{geo-fí-si-co}
\hyphenation{MCSEM}

\captionsetup{labelsep=period,font=small,justification=justified,labelfont=md,format=plain,labelsep=endash}

\renewcommand{\imprimircapa}{
\begin{capa}
	\begin{figure}[!t]
		\begin{center}
			\includegraphics[scale=1]{images/logo.png}
		\end{center} 
	\end{figure}
			\begin{center}
				{\ABNTEXchapterfont\bfseries{SERVIÇO NACIONAL DE APRENDIZAGEM COMERCIAL - SENAC \\ TECNOLOGIA EM ANÁLISE E DESENVOLVIMENTO DE SISTEMAS \\ CONCEITOS DE COMPUTAÇÃO 1} }
		\end{center}

\center

\vspace*{1cm}
{\ABNTEXchapterfont\bfseries\large\MakeUppercase{\imprimirautor}}\\
\vspace*{2cm}
{\ABNTEXchapterfont\bfseries\large\imprimirtitulo}\\
\vspace*{2cm}{\ABNTEXchapterfont\bfseries\large{PRODUÇÃO TEXTUAL INDIVIDUAL (PTI)}}
\vspace*{\fill}\\
{\large\MakeUppercase{\imprimirlocal}}
\par
{\large\imprimirdata}
\vspace*{1cm}
\end{capa}
}


\makeatletter
\renewcommand{\folhaderostocontent}{
\begin{center}
	\vspace*{1cm}
	{\ABNTEXchapterfont\bfseries\large\MakeUppercase{\imprimirautor}}\\
	\vspace*{\fill}\vspace*{\fill}
	{\ABNTEXchapterfont\bfseries\large\imprimirtitulo}\\
	 \vspace{\baselineskip}
	{\ABNTEXchapterfont\bfseries\large{PRODUÇÃO TEXTUAL INDIVIDUAL (PTI)}}
	\vspace{\baselineskip}
	\vspace{\baselineskip}
	\vspace*{\fill}
	\abntex@ifnotempty{\imprimirpreambulo}{
	\hspace{5cm}
	\begin{minipage}{10cm}
	\SingleSpacing
	\imprimirpreambulo\\ \\
	{Professor: Anderson Aparecido Alves da Silva.}\\
	\end{minipage}
	\vspace*{\fill}
	}
	%\abntex@ifnotempty{\imprimircoorientador}{
	%{\large Tutor: \imprimircoorientador}
	%}
	\vspace*{\fill}
	{\large\imprimirlocal}
	\par
	{\large\imprimirdata}
	\vspace*{1cm}
\end{center}

}

\makeatother

\begin{document}

\imprimircapa

\imprimirfolhaderosto*


%\renewcommand{\listfigurename}{\fontsize{12pt}{\baselineskip}\textbf{LISTA DE ILUSTRAÇÕES}}
%\pdfbookmark[0]{\listfigurename}{lof}
%\listoffigures*

\cleardoublepage

\tableofcontents*

\cleardoublepage
 
\mainmatter

%% Inclusão dos capítulos ao documento principal
\chapter{Produção Textual Individual (PTI)}
\label{pti}

A arquitetura de um processador
pode ser entendida como um contrato entre o hardware e o software,
esta se refere a um especificação funcional do processador que determina
como este irá se comportar, quais instruções possui
e quais instruções é capaz de executar. No caso da arquitetura ARM, esta também expõe um conjunto
de instruções comuns para os desenvolvedores para permitir diferentes implementações da arquitetura
em difrentes dispositivos \cite{arm}.

Neste ínterim se destacam as arquiteturas de processador ARM e X86, cuja
diferença se dá principalmente na complexidade do seu conjunto de instruções: O X86 é baseado na
Complex Instruction Set Computer (CISC) e o ARM é baseado na Reduced Instruction Set Computer (RISC).
Em resumo, enquanto os processadores ARM (RISC) utilizam várias instruções simples para execução da mesma tarefa \cite{tndbrasil}, a arquitetura CISC transfere a complexidade do software para o hardware reduzindo 
o número de instruções por programa na
equação de desempenho do processador \cite{bhandarkar}:

\begin{equation}
 \frac{tempo}{programa} = \frac{instruções}{programa} \times \frac{ciclos}{instrução} \times \frac{tempo}{ciclo}
\end{equation}

Os processadores ARM possuem a vantagem de consumir menos energia, maior eficiência térmica (menor dissipação de calor) e são
mais simples do que os X86. Por isso, esta arquitetura é ideal para dispositivos compactos, como smartphones e tablets,
já que estes não podem contar com sistemas complexos de dissipação de calor e utilizam processadores
de menor desempenho \cite{canaltech}.

Assim, empresas que conseguirem subtituir os processadores X86 pelos ARM podem obter algumas
vantagens: Como o baixo consumo de energia e maior autonomia da bateria, no caso de nootebooks.
Uma destas empresas é a Apple, que busca substituir os processadores X86 para ARM nos computadores Mac \cite{super}.
A principal desvantagem desta substituição é que os programas e aplicativos
precisam ser convertidos do conjunto de instruções de X86 para o conjunto da arquitetura ARM,
antes desta conversão haverão problemas de compatibilidade.
Uma solução seria utilizar emuladores para rodar programas X86 em uma plataforma com
processador ARM, oque pode não ser viável e ter o efeito
adverso de diminuir a autonomia da bateria e redução da performance do computador.

%\include{cre} % Método CRE fundamentos teóricos
%\include{vfsa} % Otimização global VFSA
%\include{pef} % Teoria intepolação com filtros PEF
%\include{modeling} % Resultado da modelagem
%\include{interpolacao} % Resultados da Interpoção com PEF
%\include{empilhamento} % Resultados do empilhamento cre
%\include{velocidades} % Proposta de uma metodologia para inversão do modelo de velocidades
%\include{cronograma} % cronograma de atividades e entrega do trabalho
%\include{conclusao} % conclusão discutindo os resultados obtidos e esperados

\bookmarksetup{startatroot}

\bibliography{mybib}

%% Inicia os apêndices
%% ---
%\begin{apendicesenv}

%% Imprime uma página indicando o início dos apêndices
%\partapendices
%\newpage\null\thispagestyle{empty}
% \begin{center}
% \vspace{10cm}
% {\large\textbf{APÊNDICES}}
% \end{center}
% \newpage

%\input{apendice_tempo_transito_cre}
% \input{apendiceB}
% \input{apendiceC}
% \input{apendiceD}
% \input{MT}
%
%\end{apendicesenv}

%% Inicia os anexos
%\begin{anexosenv}
%\partanexos
%\include{AnexoA}

%\end{anexosenv}

\cleardoublepage
\phantomsection 
\printindex

\end{document}
