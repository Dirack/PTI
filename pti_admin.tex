\chapter{Produção Textual Individual (PTI)}
\label{pti}

Os fundamentos da teoria da administração científica
criada por Taylor continuam imprescindíveis às organizações \cite{matosfilho}.
Pois, estas organizações ainda se defrontam com
a questão da busca pela maximização da produtividade do trabalhador e redução de custos, dos quais depende
a sua participação em mercados cada vez mais competitivos.
Já a teoria clássica de Fayol estabeleceu princípios gerais da administração que são observados nas organizações atuais \cite{rsh},
e as funções do administrador: organizar, planejar,coordenar, comandar e controlar \cite{mota,trag,chiavenato}.

Estas perspectivas baseadas na relação entre a prosperidade econômica e a produtividade dos trabalhadores
através da racionalização do trabalho \cite{maximiano} persistem 
não sem críticas e apresentação de seus pontos negativos, como a alienação e desmotivação gerada
pela hierarquia rígida e a separação entre concepção e execução \cite{matospires}.

Como oposição à estas concepções, essencialmente prescritivas e normativas, que não consideravam o conteúdo psicológico e social das organizações \cite{chiavenato}, surgiu a teoria humanística
da administração, mais focada nas relações humanas e nos aspectos sociais e emocionais do trabalhador \cite{chiavenato}.

Há uma mescla destas abordagens nas organizações atuais, estas
se valem da busca pela maximização da produtividade e otimização dos
processos, e adotam sistemas de gestão participativa e flexível. No contexto
do desenvolvimento de software temos o exemplo das \textit{metodologias ágeis},
resumidas nos princípios:
Indivíduos e interações mais que processos e ferramentas,
Software em funcionamento mais que documentação abrangente,
Colaboração com o cliente mais que negociação de contratos,
Responder a mudanças mais que seguir um plano \cite{agil}.

Outro exemplo é a metodologia SCRUM \cite{scrum}:
Esta utiliza feedback dos clientes, equipes multidiciplinares e auto-organizadas para atingir os objetivos
do projeto, não há um líder, apenas papéis diferentes. Todavia, também
podemos afirmar que apesar da flexibilidade na tomada de decisão e o foco
mais participativo, o SCRUM utiliza o acompanhamento da
produtividade através de gráficos de backlog, sprints, reuniões diárias e definições
de funções específicas preservando também os fundamentos
das escolas de Taylor e Fayol.
