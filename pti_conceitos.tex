\chapter{Produção Textual Individual (PTI)}
\label{pti}

O estudo referenciado no enunciado desta atividade de Produção Textual Individual (PTI)
é uma revisão sistemática e meta-análise
de outros 56 estudos que serviram de base de dados da pesquisa.
Estes estudos são provenientes das
instituições:
Medline (Ovid), Embase, Cochrane CENTRAL, WHO
International Clinical Trials Registry Platform,
Clinicaltrials.gov e várias outras fontes relevantes \cite{estudo}.

A importância de uma revisão sistemática deste tipo consiste na sua capacidade de integrar
a informação existente na literatura de maneira eficiente para os profissionais e
responsáveis pelas políticas de saúde \cite{mulrow}. Isto é feito
através de uma pesquisa planejada para sumarizar vários estudos
médicos respondendo uma questão clínica específica e de técnicas de
metanálise, que consiste na combinação
quantitativa dos dados dos estudos a partir de métodos estatísticos \cite{meta}.


Neste ínterim, o principal objetivo desta revisão sistemática é
avaliar se o consumo de adoçantes
gera algum benefício para
adultos e crianças saudáveis ou com obesidade ou sobrepeso. Oque pode ser
resumido na seguinte questão-problema: Existe alguma associação entre o consumo de adoçantes
e o emagrecimento ou algum outro benefício para a saúde?
Um objetivo menor do estudo é buscar evidências de efeitos colaterais ou eventos adversos,
pois a comunidade médica apresenta a preocupação com a hipótese dos adoçantes serem causa
de cáries e diabetes \cite{estudo}. 

A conclusão desta revisão sistemática é
que a partir dos estudos analisados não é possível observar alguma correlação entre
emagrecimento ou benefícios e malefícios à saúde e o consumo de adoçantes.
Pois, não foi observada diferença significativa
entre o grupo exposto ao consumo de adoçantes e não exposto (grupo de controle, que não consumiu adoçantes em sua dieta durante o período do estudo) em relação aos parâmetros observados na pesquisa.
Também foi observado em alguns poucos estudos desta revisão
que alguns participantes obtiveram pequenos benefícios: Como
redução no Índice de Massa Corpórea (IMC) e no nível de açúcar no sangue.
Todavia, esta observação foi feita em poucos estudos e em um limitado número
de participantes, por isso esta é considerada uma evidência fraca do possível benefício
do consumo de adoçantes para a saúde \cite{estudo}.
