\chapter{Produção Textual Individual (PTI)}
\label{pti}

A arquitetura de um processador
pode ser entendida como um contrato entre o hardware e o software,
esta se refere a um especificação funcional do processador que determina
como este irá se comportar, quais instruções possui
e quais instruções é capaz de executar. No caso da arquitetura ARM, esta também expõe um conjunto
de instruções comuns para os desenvolvedores para permitir diferentes implementações da arquitetura
em difrentes dispositivos \cite{arm}.

Neste ínterim se destacam as arquiteturas de processador ARM e X86, cuja
diferença se dá principalmente na complexidade do seu conjunto de instruções: O X86 é baseado na
Complex Instruction Set Computer (CISC) e o ARM é baseado na Reduced Instruction Set Computer (RISC).
Em resumo, enquanto os processadores ARM (RISC) utilizam várias instruções simples para execução da mesma tarefa \cite{tndbrasil}, a arquitetura CISC transfere a complexidade do software para o hardware reduzindo 
o número de instruções por programa na
equação de desempenho do processador \cite{bhandarkar}:

\begin{equation}
 \frac{tempo}{programa} = \frac{instruções}{programa} \times \frac{ciclos}{instrução} \times \frac{tempo}{ciclo}
\end{equation}

Os processadores ARM possuem a vantagem de consumir menos energia, maior eficiência térmica (menor dissipação de calor) e são
mais simples do que os X86. Por isso, esta arquitetura é ideal para dispositivos compactos, como smartphones e tablets,
já que estes não podem contar com sistemas complexos de dissipação de calor e utilizam processadores
de menor desempenho \cite{canaltech}.

Assim, empresas que conseguirem subtituir os processadores X86 pelos ARM podem obter algumas
vantagens: Como o baixo consumo de energia e maior autonomia da bateria, no caso de nootebooks.
Uma destas empresas é a Apple, que busca substituir os processadores X86 para ARM nos computadores Mac \cite{super}.
A principal desvantagem desta substituição é que os programas e aplicativos
precisam ser convertidos do conjunto de instruções de X86 para o conjunto da arquitetura ARM,
antes desta conversão haverão problemas de compatibilidade.
Uma solução seria utilizar emuladores para rodar programas X86 em uma plataforma com
processador ARM, oque pode não ser viável e ter o efeito
adverso de diminuir a autonomia da bateria e redução da performance do computador.
