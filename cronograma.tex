\chapter{CRONOGRAMA}
\label{cap9:cronograma}


\section{Etapas concluídas}

A seguir a descrição das etapas concluídas e o cronograma:

  \begin{enumerate}
   \item  Testes preliminares do algoritmos, Pesquisa e fundamentação teórica (09/2017 - 08/2019): Produção de testes 
   de implementação
   dos algoritmos utilizados e primeiras versões dos algoritmos originais do Autor.
   Pesquisa das principais referências teóricas e estabelecimento do tema central da tese.
   \item   Modelagem Kirchhoff (09/2019): Utilização do algoritmo de modelagem 
   do pacote Madagascar\footnote{Madagascar é um pacote
   de procesamento sísmico 'open source' disponível em \url{http://www.ahay.org/wiki/Main_Page}.}
   \textit{sfkirmod} para produzir os dados do modelo do refletor
   gaussiano.
    \item Obtenção dos parâmetros do SRC utilizando o VFSA (09/2019): Programa \textit{sfvfsacrenh} escrito  pelo Autor
    em linguagem C e adaptado para o pacote Madagascar
   baseado no algoritmo Very Fast Simulated Aneeling \cite{ingber}. O algoritmo ajusta a superfície de tempo de trânsito
   do SRC não hiperbólico (Equação \ref{eq:2.4}) aos dados modelados. Os parâmetros do SRC que produzem o melhor ajuste são
   os parâmetros otimizados.
    \item  Interpolação do cubo de dados com FPE (09/2019): Interpolação das seções de
    afastamento constante extraídas dos dados modelados
    (chamado ``cubo de dados'') a partir de Filtros Adaptativos de Predição de Erro
    com os programas \textit{sfapef} e \textit{sfmiss4} do pacote Madagascar.
    Esta interpolação permite a discretização
    suficiente dos traços no domínio do PMC para possibilitar a correta amostragem das famílias ERC.
    \item  Cálculo das trajetórias ERC (09/2019): Programa \textit{sfcretrajec} desenvolvido pelo Autor em linguagem C 
    e adaptado para o pacote Madagascar para
    o cálculo das trajetorias ERC baseado na Equação \ref{eq:2.1}.
     \item Obtenção das famílias ERC (09/2019): Programa \textit{sfgetcregather} desenvolvido pelo Autor em linguagem C 
     e adaptado para o pacote Madagascar para a
     determinação dos traços sísmicos do cubo de dados que estão sobre as trajetorias ERC previamente calculadas na etapa anterior.
     Estes traços formam as famílias ERC.
    \item  Cálculo das curvas de empilhamento ERC (10/2019): Programa \textit{sfgetcretimecurve} desenvolvido pelo Autor 
    em linguagem C e adaptado 
    para o pacote Madagascar para a determinação das curvas de tempo de trânsito ERC com auxílio das 
    Equações \ref{eq:2.3}-\ref{eq:2.4}.
     \item Paralelização do algoritmo com scons (12/2019): Utilização de técnicas de computação paralela para 
     melhorar o desempenho
     dos algoritmos desenvolvidos. O pacote Madagascar permite a Paralelização dos processos realizados a partir da execução
     com o comando \textit{scons -j\#} onde '\#' representa o número de núcleos utilizados.
    \item  Empilhamento e seção empilhada ERC (12/2019): Programa \textit{sfcrestack} 
    desenvolvido pelo Autor em linguagem C e adaptado 
    para o pacote Madagascar para a obtenção da seção  empilhada ERC.
  \end{enumerate}

    \begin{table}[H]
      \caption{Cronograma de trabalho das etapas já concluídas em cada mês do ano de 2019.}
      \centering
      
      \begin{tabular}{|p{6cm}|c|c|c|c|c|}%{|c|c|c|c|c|c|}

     \hline
      \textbf{Etapas concluídas} & 08/19 & 09/19 & 10/19 & 11/19 & 12/19 \\ \hline
      Pesquisa e fundamentação teórica & x & & &  & \\ \hline
      Modelagem Kirchhoff & & x & &  & \\ \hline
      Obtenção dos parâmetros do SRC utilizando o VFSA & & x & &  & \\ \hline
      Interpolação do cubo de dados com FPE & & x & &  & \\ \hline
      Cálculo das trajetórias ERC & & x & &  & \\ \hline
      Obtenção das famílias ERC & & x & &  & \\ \hline
      Cálculo das curvas de empilhamento ERC & & & x & & \\ \hline
      Paralelização do algoritmo com scons & & & & & x \\ \hline
      Empilhamento e seção empilhada ERC & & & &  & x \\
      \hline
      
      \end{tabular}
  \end{table}
  
\section{Etapas ainda não concluídas}

A seguir a descrição das atividades ainda não realizadas e o cronograma:

  \begin{enumerate}
   \item Qualificação da tese
   \item Inversão do modelo de velocidades
   \item Defesa de tese
  \end{enumerate}
  
   \begin{table}[H]
      \caption{Cronograma de trabalho até o prazo final da defesa de tese.}
      \centering
      
      \begin{tabular}{|c|c|c|}

      \hline
      \textbf{Etapas ainda não concluídas} & 2 & 3 \\ \hline
      Qualificação da tese & x & x \\ \hline
      Modelo de velocidades inicial (NIP tomografia) & x & x \\ \hline
      Simulador de resposta de difrações & x & x \\ \hline
      Algoritmo de focalização de difrações & x & x \\ \hline
      Inversão do modelo de velocidades VFSA & x & x \\ \hline
      Defesa de tese & x & x  \\
      \hline
      
      \end{tabular}
  \end{table}
  

Algumas observações: A data da apresentação para o Comitê de Avaliação de Tese é o dia 01 de Agosto de 2021.
A data da apresentação para a Banca Avaliadora dependerá das sugestões do
Comitê de Avaliação de Tese. A Previsão para defesa é entre os meses de Agosto de
2021 e Setembro de 2021.

