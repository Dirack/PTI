\chapter{INTRODUÇÃO}
\label{intro}

A equação do sobretempo normal \cite{dix} é uma aproximação de tempo de trânsito de reflexão válida para pequenos afastamentos
entre os pares fonte receptor na superfície de registro em uma aquisição sísmica. Esta equação foi desenvolvida para modelos de
multicamadas planas horizontais. 
A equação de sobretempo normal é uma aproximação em série de Taylor de segunda ordem para o tempo de trânsito, por isto é
também chamada de aproximação hiperbólica. Todavia, esta aproximação diverge do tempo de trânsito analítico
em grandes afastamentos.

Aproximações de tempo de trânsito não hiperbólicas foram desenvolvidas na literatura,
com o intuito de estender a região de convergência das
aproximações de tempo de trânsito no domínio do afastamento: Estas aproximações utilizam mais de dois termos para aumentar a 
acurácia da análise de velocidades e a correção de sobretempo normal. Cada uma destas aproximações terá
as suas limitações na estimativa das velocidades e na relação afastamento profundidade.

O empilhamento convencional, utilizando as aproximações de sobretempo normal hiperbólicas ou não hiperbólicas, é estendido
para o domínio do Ponto Médio Comun (PMC) a partir do empilhamento Superfície de Reflexão Comum (SRC): 
O empilhamento é realizado sobre uma superfície de tempo de
trânsito no domínio do meio afastamento $h$ e na vizinhança de um PMC central $m_0$, a partir de
três parâmetros ($R_N$, $R_{NIP}$ e $\beta_0$) dados para cada $m_0$.
O empilhamento SRC possui uma aproximação de tempo de trânsito hiperbólica \cite{jager}, e várias aproximações
de tempo de trânsito SRC foram propostas com o objetivo de estender a região de convergência desta aproximação:
As aproximações do SRC não hiperbólico \cite{fomel1}, SRC quarta ordem
\cite{germam}.

Um caso especial do método de empilhamento SRC, é o empilhamento por Elemento de Reflexão Comum (ERC):
O empilhamento ERC é também realizado no domínio do afastamento $h$ e na vizinhança de um PMC central $m_0$, assim como o
empilhamento SRC. Porém, o empilhamento ERC é feito sobre uma curva de tempo de trânsito de reflexão correspondente ao
conjunto de trajetórias de reflexão que possuem em comun
o mesmo ponto de incidência sobre o refletor.
Este método possui a vantagem de prover
parâmetros importantes para a construção do modelo de velocidades, e utiliza dois dos parâmetros
do método SRC ($R_{NIP}$ e $\beta_0$).
A principal desvantagem do método ERC é a interpolação: O empilhamento ERC necessita de 
traços em coordenadas de PMC $m$ e meio afastamento $h$ não coincidentes com as coordenadas
e amostragem dos dados adquiridos.

O objetivo desta pesquisa é propor uma metodologia de interpolção para o empilhamento ERC 
a partir dos parâmetros extraídos da superfície de tempo de trânsito SRC. Utilizaremos o 
algoritmo Very Fast Simulated Aneeling (VFSA) para obter os parâmetros do SRC de afastamentos nulo.
Traçaremos a trajetória ERC no plano $m, h$ para estabelecer as coordenadas dos traços sobre a curva.
Interpolaremos os dados adiquiridos utilizando os filtros adaptativos de predição de erro. E montaremos
o a família ERC para cada $m_0$ e trajetória ERC determinada na etapa anterior. Finalizadas estas etapas,
obteremos a seção empilhada ERC a partir do empilhamento sobre uma curva de tempo de trânsito ERC, determinada
para cadar par $m_0, t_0$, sobre as famílias ERC.
